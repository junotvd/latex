\setcounter{chapter}{5}
\chapter{Direct Current Circuits}
In the book:~\cite[Ch. 26]{giancoli}

\lesson{Lesson Module}{6}{Mon, Feb 24, 2025}

The battery is a source of \textbf{electromotive force (emf)}.
The emf $\varepsilon$ of a battery is the maximum potential difference that the battery can generate across its terminals.

\begin{remark}{}
    The emf provides energy, so it is not a force in the literal sense.
\end{remark}

The battery is the energy source for an electric circuit.
This allows charges to gain higher potential energy within the battery.

We assume that wires have no electrical resistance.
The positive terminal of the battery is at a higher potential than the negative terminal.

\begin{remark}{}
    However, the battery also has an \textbf{internal resistance}!
    If the internal resistance were zero, the delivered voltage would be equal to the emf.
    In a real battery, there is always some internal resistance $r$.
\end{remark}

The voltage across the terminals is
\[
    V_{\text{ab}} = \varepsilon - Ir.
\]
On the other hand, $V_{\text{ab}} = IR$ because the potential at $d$ must be equal to the potential at $a$.

The actual potential difference across the terminals is called the \textbf{terminal voltage} and depends on the current flowing in the circuit.

The voltage across the battery’s terminals is equal to the voltage across the external resistor.
This resistance is called the \textbf{load resistance} (denoted as $R$ in DC circuits).


\section{Resistors in Series}

The potential difference $V$ is distributed among the different resistors, and the total potential differences sum up due to energy conservation:
\[
    V = IR_1 + IR_2 = I(R_1 + R_2).
\]
For $n$ resistors in series, we have:
\[
    R_{\text{eq}} = R_1 + R_2 + \cdots + R_n.
\]


\section{Resistors in Parallel}

The potential difference across all resistors is the same because they are directly connected to the battery terminals.
\[
    I = I_1 + I_2 = \frac{V}{R_1} + \frac{V}{R_2} = V\left(\frac{1}{R_1}+\frac{1}{R_2}\right) = \frac{V}{R_{\text{eq}}}
\]
\[
    \implies R_{\text{eq}} = {\left(\frac{1}{R_1}+\frac{1}{R_2}\right)}^{-1} = \frac{R_1R_2}{R_1 + R_2}.
\]
For $n$ resistors, we obtain:
\[
    \frac{1}{R_{\text{eq}}} = \frac{1}{R_1} + \frac{1}{R_2} + \cdots + \frac{1}{R_n}.
\]


\section{Kirchhoff’s Rules}

For certain branched resistor circuits, it is not possible to reduce the resistances to a single equivalent resistance using only series and/or parallel combinations.
Kirchhoff’s rules, named after Gustav Kirchhoff (1824–1887), can be used to analyze such circuits.

\begin{theorem}{Kirchhoff’s First Rule}{kirchhoff1}
    The sum of currents entering a junction must be equal to the sum of currents leaving the junction:
    \[
        \sum I_{\text{in}} = \sum I_{\text{out}}.
    \]
\end{theorem}

\begin{theorem}{Kirchhoff’s Second Rule}{kirchhoff2}
    The sum of the potential differences around any closed loop in a circuit must be zero:
    \[
        \sum_{\text{closed loop}} \Delta V = 0.
    \]
\end{theorem}


\subsection{Charging an RC Circuit}

Applying \Cref{thm:kirchhoff2} to a closed loop gives:
\[
    \varepsilon - \frac{q(t)}{C} - I(t)R = 0 \implies \varepsilon - \frac{q(t)}{C} - R \frac{dq(t)}{dt} = 0.
\]
This is a differential equation for $q(t)$.
We solve it using separation of variables:
\[
    \frac{C\varepsilon}{RC} - \frac{q(t)}{RC} - \frac{dq(t)}{dt} = 0 \implies \frac{dq(t)}{C\varepsilon -q(t)} = \frac{dt}{RC}.
\]
Integrating both sides between $t=0$ and $t$ where the capacitor charge reaches $q$:
\[
    \int_0^q \frac{dq}{q-C\varepsilon} = -\frac{1}{RC}\int_0^t dt \implies \ln\left(\frac{q-C\varepsilon}{-C\varepsilon}\right) = -\frac{t}{RC}.
\]
Since for $t \to \infty$, the capacitor charge approaches $Q = C\varepsilon$, the time-dependent charge is:
\[
    q(t) = C\varepsilon\left(1-e^{-\frac{t}{RC}}\right) = Q\left(1-e^{-\frac{t}{RC}}\right).
\]
Differentiating with respect to time gives:
\[
    I(t) = \frac{\varepsilon}{R}e^{-\frac{t}{RC}} = I(t=0)e^{-\frac{t}{RC}}.
\]
We define the \textbf{time constant} $\tau$ for the RC circuit as:
\[
    \tau = RC.
\]
$\tau$ corresponds to the time required for the charge to increase from zero to 63.2% of its maximum value:
\[
    Q = C\varepsilon(1-\frac{1}{e}).
\]
During this time, the current decreases by a factor of $\frac{1}{e}$, reducing to 36.8% of its initial value.

The total energy supplied by the emf source during charging is:
\[
    U_{\text{emf}} = \int_0^{U_{\text{emf}}} dU = \int_0^Q \varepsilon\,dq = Q\varepsilon = C\varepsilon^2.
\]
From~\cite[Ch. 24]{giancoli}, the energy stored in the capacitor is:
\[
    U_C = \frac{C\varepsilon^2}{2}.
\]
This is because some energy is dissipated as heat in the resistor:
\[
    U_R = \frac{C\varepsilon^2}{2}.
\]
Thus, the total energy provided by the emf source is equally distributed between the resistor and the capacitor:
\[
    U_{\text{emf}} = U_C + U_R.
\]


\subsection{Discharging an RC Circuit}

Discharging occurs with the same time constant $\tau = RC$ as charging.
Applying \Cref{thm:kirchhoff2} to a closed loop leads to:
\[
    \frac{q(t)}{C} - I(t)R = 0, \quad \text{where} \quad I(t) = -\frac{dq(t)}{dt}.
\]
\[
    -R \frac{dq(t)}{dt} = \frac{q(t)}{C} \implies -\frac{dq(t)}{q(t)} = \frac{dt}{RC}.
\]
Integrating from $t=0$ to $t$, where charge decreases from $Q$ to $q(t)$:
\[
    \int_Q^{q(t)}\frac{dq}{q} = -\int_0^t \frac{dt}{RC} \implies \ln\left(\frac{q}{Q}\right) = -\frac{t}{RC}.
\]
Thus, the charge decays as:
\[
    q(t) = Qe^{-\frac{t}{RC}}.
\]
This results in the current:
\[
    I(t) = I(t=0)e^{-\frac{t}{RC}}.
\]

The energy originally stored in the capacitor is dissipated as heat in the resistor.
