% \usepackage[T1]{fontenc}
% \usepackage{PTSerif}
\usepackage{fontspec}
% \usepackage[bitstream-charter]{mathdesign}

\usepackage[english]{babel}

\usepackage{geometry}
\geometry{%
    paperwidth=210mm,
	paperheight=297mm,
    top=72pt,
    marginparsep=20pt,
    marginparwidth=60pt,
    head=24pt,
    textheight=690pt,
    footskip=40pt,
}


\usepackage{float}
\usepackage{ragged2e}
\usepackage{needspace}
\usepackage{xspace}

\usepackage{csquotes}
\usepackage{parskip}

\usepackage{tikz}
\usepackage{graphicx}

\usepackage{booktabs}
\usepackage{tabularx}
\usepackage{enumitem}

\usepackage{amsmath, amsfonts, mathtools, amsthm, amssymb}
\usepackage{bm, cancel}

\usepackage{microtype} %load after amsthm

% \usepackage{showframe}
% \usepackage{layout}
\usepackage{lipsum}

\usepackage[
    sorting = nyt,
    style = alphabetic
]{biblatex}

\usepackage[automark]{scrlayer-scrpage}
\usepackage{scrlayer-notecolumn}
\setkomafont{notecolumn.marginpar}{\footnotesize}


\usepackage[
    colorlinks=true,
    linkcolor=black,
    filecolor=magenta,
    citecolor=black,
    urlcolor=cyan,
    pdfpagemode=FullScreen,
]{hyperref}

\usepackage[english]{cleveref}
\crefname{equation}{}{}


\renewcommand\geq{\geqslant}
\renewcommand\leq{\leqslant}

\let\epsilon\varepsilon
% \let\phi\varphi

\pagestyle{scrheadings}
\lehead{\thepage}
\rohead{\thepage}
\rehead{\MakeUppercase{\sffamily\textls\leftmark}}
\lohead{\MakeUppercase{\sffamily\textls\rightmark}}
\lofoot{}
\refoot{}
\lefoot{}
\rofoot{}
\chead{}
\cfoot{}

\usepackage{xifthen}
\makeatletter
\def\@les{}%
\newcommand{\lesson}[3]{%
    \def\@lesson{#1 #2}%
    {\def\@lesson{#1 #2}}
    \subsection*{\@lesson}
    \makenote{\small\textsf{\mbox{#3}}}
}
\makeatother

\usepackage{import}
\usepackage{pdfpages}
\usepackage{transparent}
\newcommand{\incfig}[2][1]{%
    \def\svgwidth{#1\columnwidth}
    \import{./figures/}{#2.pdf_tex}
}
% \pdfsuppreswarningpagegroup=1


\usepackage[most]{tcolorbox}

\tcbset{%
    parbox = false,
}
\tcbset{%
    boxstyle/.style = {%
        parbox=false,
        enhanced,
        sharp corners,
        colframe = black,
        coltitle = black,
        colback = white,
        fonttitle = \sffamily\bfseries,
        after title = {.},
        description font = \mdseries,
        attach boxed title to top left = {xshift=2mm, yshift=-2.5mm},
        separator sign none,
        description delimiters parenthesis,
        boxed title style = {%
            colframe = white,
            colback = white,
            sharp corners,
        },
        boxrule = 0.5pt,
    },
    proofstyle/.style = {%
        parbox=false,
        enhanced,
        blanker,
        breakable,
        coltitle = black,
        fonttitle = \sffamily\bfseries,
        description font = \mdseries,
        attach title to upper = {.\;},
        left = 5mm,
        top = 1mm,
        borderline west = {0.5pt}{0pt}{black},
        description delimiters parenthesis,
        separator sign none,
    },
    plainstyle/.style = {%
        parbox=false,
        enhanced,
        sharp corners,
        boxrule = 0pt,
        boxsep = 0mm,
        left = 0mm,
        breakable,
        attach title to upper = {.\:},
        % after title = {\;},
        description font = \mdseries,
        coltitle = black,
        colback = white,
        colframe = white,
        fonttitle = \sffamily\bfseries,
        description delimiters parenthesis,
        separator sign none,
    },
    wrongstyle/.style = {%
        colback = red,
    },
}


\newtcbtheorem[auto counter,number within=chapter,crefname={definition}{definitions}]%
    {definition}{Definition}{boxstyle}{def}

\newtcbtheorem[use counter from=definition,crefname={axiom}{axioms}]%
    {axiom}{Axiom}{boxstyle}{axi}

\newtcbtheorem[use counter from=definition,crefname={property}{properties}]%
    {property}{Property}{plainstyle,breakable}{prp}

\newtcbtheorem[use counter from=definition,crefname={lemma}{lemmas}]%
    {lemma}{Lemma}{boxstyle,fontupper=\itshape}{lem}

\newtcbtheorem[use counter from=definition,crefname={theorem}{theorems}]%
    {theorem}{Theorem}{boxstyle,fontupper=\itshape}{thm}

\newtcbtheorem[use counter from=definition]%
    {statement}{Statement}{boxstyle,fontupper=\itshape}{stmt}

\newtcbtheorem[use counter from=definition,crefname={proposition}{propositions}]%
    {proposition}{Proposition}{boxstyle,fontupper=\itshape}{prop}

\newtcbtheorem[use counter from=definition,crefname={corollary}{corollaries}]%
    {corollary}{Corollary}{boxstyle,breakable}{cor}

\newtcbtheorem[no counter,crefname={proof}{proofs}]%
    {proof}{Proof}{plainstyle,breakable,top=0mm}{prf}

\newtcbtheorem[use counter from=definition,crefname={example}{examples}]%
    {example}{Example}{plainstyle,breakable}{ex}

\newtcbtheorem[use counter from=definition,crefname={application}{applications}]%
    {application}{Application}{plainstyle,breakable}{app}

\newtcbtheorem[auto counter,number within=chapter,crefname={exercise}{exercises}]%
    {exercise}{Exercise}{plainstyle,breakable}{exer}

\AtEndEnvironment{proof}{\qed}
\AtEndEnvironment{example}{\null\hfill$\diamond$}

\newtcolorbox{solution}{plainstyle,title=Solution,top=0mm}
\newtcolorbox{answer}{plainstyle,title=Answer,top=0mm}
\newtcolorbox{elaboration}{plainstyle,title=Elaboration,top=0mm}
\newtcolorbox{remark}{plainstyle,title=Remark}
\newtcolorbox{hint}{plainstyle,title=Hint,top=0mm,fonttitle=\itshape,fontupper=\itshape,top=0mm}
\newtcolorbox{reminder}{plainstyle,title={Reminder}}
\newtcolorbox{intermezzo}{plainstyle,title=Intermezzo}
\newtcolorbox{notation}{plainstyle,title=Notation}
\newtcolorbox{convention}{plainstyle,title=Convention}
\newtcolorbox{interpretation}{plainstyle,title=Interpretation}
\newtcolorbox[auto counter]{question}{plainstyle,title={Question~\thetcbcounter}}


\author{Junot Van Dijck}
