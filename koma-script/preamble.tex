\usepackage{fontspec}

\usepackage[main=dutch,provide=*]{babel}

\usepackage{geometry}
\geometry{%
    paperwidth=210mm,
	paperheight=297mm,
    top=72pt,
    marginparsep=20pt,
    marginparwidth=60pt,
    head=24pt,
    textheight=690pt,
    footskip=40pt,
}


\usepackage{float}
\usepackage{ragged2e}
\usepackage{needspace}
\usepackage{xspace}

\usepackage{csquotes}
\usepackage{parskip}

\usepackage{tikz}
\usepackage{graphicx}

\usepackage{booktabs}
\usepackage{tabularx}
\usepackage{enumitem}

\usepackage{amsmath, amsfonts, mathtools, amsthm, amssymb}
\usepackage{bm, cancel}

\usepackage{microtype} %load after amsthm

% \usepackage{showframe}
% \usepackage{layout}
\usepackage{lipsum}

\usepackage[
    sorting = nyt,
    style = alphabetic
]{biblatex}

\usepackage[automark]{scrlayer-scrpage}
\usepackage{scrlayer-notecolumn}
\setkomafont{notecolumn.marginpar}{\footnotesize}


\usepackage[
    colorlinks=true,
    linkcolor=black,
    filecolor=magenta,
    citecolor=black,
    urlcolor=cyan,
    pdfpagemode=FullScreen,
]{hyperref}

\usepackage[dutch]{cleveref}
\crefname{equation}{}{}
\Crefname{equation}{Vergelijking}{Vergelijkingen}


\renewcommand\geq{\geqslant}
\renewcommand\leq{\leqslant}

\let\epsilon\varepsilon
% \let\phi\varphi

\DeclareMathOperator{\dom}{dom}
\DeclareMathOperator{\bld}{bld}


\pagestyle{scrheadings}
\lehead{\thepage}
\rohead{\thepage}
\rehead{\MakeUppercase{\sffamily\textls\leftmark}}
\lohead{\MakeUppercase{\sffamily\textls\rightmark}}
\lofoot{}
\refoot{}
\lefoot{}
\rofoot{}
\chead{}
\cfoot{}

\usepackage{xifthen}
\makeatletter
\def\@les{}%
\newcommand{\les}[3]{%
    \def\@les{#1 #2}%
    {\def\@les{#1 #2}}
    \subsection*{\@les}
    % \marginpar{\small\textsf{\mbox{#3}}}
    \makenote{\small\textsf{\mbox{#3}}}
}
\makeatother

\usepackage{import}
\usepackage{pdfpages}
\usepackage{transparent}
\newcommand{\incfig}[2][1]{%
    \def\svgwidth{#1\columnwidth}
    \import{./figures/}{#2.pdf_tex}
}
% \pdfsuppreswarningpagegroup=1


\usepackage[most]{tcolorbox}

\tcbset{%
    parbox = false,
}
\tcbset{%
    boxstyle/.style = {%
        parbox=false,
        enhanced,
        sharp corners,
        colframe = black,
        coltitle = black,
        colback = white,
        fonttitle = \sffamily\bfseries,
        after title = {.},
        description font = \mdseries,
        attach boxed title to top left = {xshift=2mm, yshift=-2.5mm},
        separator sign none,
        description delimiters parenthesis,
        boxed title style = {%
            colframe = white,
            colback = white,
            sharp corners,
        },
        boxrule = 0.5pt,
    },
    proofstyle/.style = {%
        parbox=false,
        enhanced,
        blanker,
        breakable,
        coltitle = black,
        fonttitle = \sffamily\bfseries,
        description font = \mdseries,
        attach title to upper = {.\;},
        left = 5mm,
        top = 1mm,
        borderline west = {0.5pt}{0pt}{black},
        description delimiters parenthesis,
        separator sign none,
    },
    plainstyle/.style = {%
        parbox=false,
        enhanced,
        sharp corners,
        boxrule = 0pt,
        boxsep = 0mm,
        left = 0mm,
        right=0mm,
        breakable,
        attach title to upper = {.\:},
        % after title = {\;},
        description font = \mdseries,
        coltitle = black,
        colback = white,
        colframe = white,
        fonttitle = \sffamily\bfseries,
        description delimiters parenthesis,
        separator sign none,
    },
    wrongstyle/.style = {%
        colback = red,
    },
}


\newtcbtheorem[auto counter,number within=chapter,crefname={definitie}{definities}]%
    {definitie}{Definitie}{boxstyle}{def}

\newtcbtheorem[use counter from=definitie,crefname={axioma}{axiomas}]%
    {axioma}{Axioma}{boxstyle}{axi}

\newtcbtheorem[use counter from=definitie,crefname={eigenschap}{eigenschappen}]%
    {eigenschap}{Eigenschap}{plainstyle,breakable}{eig}

\newtcbtheorem[use counter from=definitie,crefname={lemma}{lemmas}]%
    {lemma}{Lemma}{boxstyle,fontupper=\itshape}{lem}

\newtcbtheorem[use counter from=definitie,crefname={stelling}{stellingen}]%
    {stelling}{Stelling}{boxstyle,fontupper=\itshape}{stel}

\newtcbtheorem[use counter from=definitie]%
    {bewering}{Bewering}{boxstyle,fontupper=\itshape}{bewr}

\newtcbtheorem[use counter from=definitie,crefname={propositie}{proposities}]%
    {propositie}{Propositie}{boxstyle,fontupper=\itshape}{prop}

\newtcbtheorem[use counter from=definitie,crefname={gevolg}{gevolgen}]%
    {gevolg}{Gevolg}{boxstyle,breakable}{gev}

\newtcbtheorem[no counter,crefname={bewijs}{bewijzen}]%
    {bewijs}{Bewijs}{plainstyle,breakable,top=0mm}{bew}

\newtcbtheorem[use counter from=definitie,crefname={voorbeeld}{voorbeelden}]%
    {voorbeeld}{Voorbeeld}{plainstyle,breakable}{vb}

\newtcbtheorem[use counter from=definitie,crefname={toepassing}{toepassingen}]%
    {toepassing}{Toepassing}{plainstyle,breakable}{tp}

\newtcbtheorem[auto counter,number within=chapter,crefname={oefening}{oefeningen}]%
    {oefening}{Oefening}{plainstyle,breakable}{oef}

\AtEndEnvironment{bewijs}{\qed}
\AtEndEnvironment{voorbeeld}{\null\hfill$\diamond$}


\newtcolorbox{oplossing}{plainstyle,title=Oplossing,top=0mm}
\newtcolorbox{antwoord}{plainstyle,title=Antwoord,top=0mm}
\newtcolorbox{uitwerking}{plainstyle,title=Uitwerking,top=0mm}
\newtcolorbox{opmerking}{plainstyle,title=Opmerking}
\newtcolorbox{hint}{plainstyle,title=Hint,top=0mm,fonttitle=\itshape,fontupper=\itshape,top=0mm}
\newtcolorbox{herinner}{plainstyle,title={Herinner}}
\newtcolorbox{intermezzo}{plainstyle,title=Intermezzo}
\newtcolorbox{notatie}{plainstyle,title=Notatie}
\newtcolorbox{conventie}{plainstyle,title=Conventie}
\newtcolorbox{interpretatie}{plainstyle,title=Interpretatie}
\newtcolorbox[auto counter]{vraag}{plainstyle,title={Vraag~\thetcbcounter}}



\author{Junot Van Dijck}
