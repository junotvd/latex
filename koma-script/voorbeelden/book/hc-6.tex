\setcounter{chapter}{5}
\chapter{Gelijkstroomketens}
In het boek:~\cite[H26]{giancoli}

\les{Lesmodule}{6}{ma 24 feb 2025}

De batterij is een bron van \textbf{elektromotorische kracht (emk)}.
De emk $\varepsilon$ van een batterij is het maximale potentiaalverschil dat de batterij kan opwekken via de polen.

\begin{opmerking}{}
    De emk levert energie, het gaat dus niet om een kracht in de letterlijke betekenis
\end{opmerking}

De batterij is de bron van energie voor een elektrish circuit.
Hierdoor verkrijgen ladingen een hogere potentiële energie in de batterij.

We nemen aan dat draden geen elektrische weerstand hebben.
De positieve pool van de batterij bevindt zich op een hogere potentiaal dan de negatieve pool.
\begin{opmerking}{}
    De batterij heeft echter ook een \textbf{inwendige weerstand}!
    Als de inwendige weerstand gelijk aan nul zou zijn, dan zou de geleverde spanning gelijk zijn aan de emk.
    In een echte batterij is er steeds een inwendige weerstand $r$.
\end{opmerking}

De spanning over de polen is
\[
    V_{\text{ab}} = \varepsilon - Ir.
\]
Anderzijds is $V_{\text{ab}} = IR$ want de potentiaal in $d$ moet gelijk zijn aan de potentiaal in $a$.

Het werkelijke potentiaalverschil tussen de polen is de \textbf{klemspanning} en hangt af van de stroom die in het circuit vloeit.

De spanning over de polen van de batterij is gelijk aan de spanning over de externe weerstand.
Deze spanning is de \textbf{belastingsweerstand} (in gelijkstroomketens aangeduid met $R$).


\section{Weerstanden in serie}

Het potentiaalverschil $V$ wordt verdeeld over de verschillende weerstanden en de potentiaalverschillen tellen we op wegens het behoud van energie:
\[
    V = IR_1 + IR_2 = I(R_1 + R_2).
\]
Voor $n$ weerstanden in serie geldt
\[
    R_{\text{eq}} = R_1 + R_2 + \cdots + R_n.
\]


\section{Weerstanden in parallel}

Het potentiaalverschil over alle weerstanden is hetzelfde omdat ze allen direct aan de polen van de batterij worden gekoppeld.
\[
    I = I_1 + I_2 = \frac{V}{R_1} + \frac{V}{R_2} = V\left(\frac{1}{R_1}+\frac{1}{R_2}\right) = \frac{V}{R_{\text{eq}}}
\]
\[
    \implies R_{\text{eq}} = {\left(\frac{1}{R_1}+\frac{1}{R_2}\right)}^{-1} = \frac{R_1R_2}{R_1 + R_2}.
\]
Voor $n$ weerstanden krijgen we dan
\[
    \frac{1}{R_{\text{eq}}} = \frac{1}{R_1} + \frac{1}{R_2} + \cdots + \frac{1}{R_n}.
\]


\section{De regels van Kirchhoff}

Voor bepaalde, vertakte circuits met weerstanden is het niet mogelijk om deze weerstanden te reduceren tot één enkele equivalente weerstand door te steunen op serie- en/of parallelschakelingen.
De regels van Kirchhoff, naar Gustav Kirchhoff, 1824 – 1887, kunnen dan gebruikt worden om toch tot een oplossing te komen.

\begin{stelling}{Eerste regel van Kirchhoff}{kicrchhoff1}
    De som van stromen die een vertakking binnenkomen moet gelijk zijn aan de som van de stromen die de vertakking verlaten:
    \[
        \sum I_{\text{in}} = \sum I_{\text{out}}.
    \]
\end{stelling}

\begin{stelling}{Tweede regel van Kirchhoff}{kirchhoff2}
    De som van de potentiaalverschillen over de elementen die deel uitmaken van een gesloten lus moet nul zijn:
    \[
        \sum_{\text{gesloten lus}} \Delta V = 0.
    \]
\end{stelling}


\subsection{Opladen van een RC-keten}

\Cref{stel:kirchhoff2} voor een gesloten lus geeft
\[
    \varepsilon - \frac{q(t)}{C} - I(t)R = 0 \implies \varepsilon - \frac{q(t)}{C} - R \frac{dq(t)}{dt} = 0.
\]
Dit is een differentiaalvergelijking voor $q(t)$.
We lossen op door scheiden van de veranderlijke:
\[
    \frac{C\varepsilon}{RC} - \frac{q(t)}{RC} - \frac{dq(t)}{dt} = 0 \implies \frac{dq(t)}{C\varepsilon -q(t)} = \frac{dt}{RC}.
\]
Integreer nu beiden leden tussen $t=0$ en $t$ waarop de condensator de waarde $q$ bereikt:
\[
    \int_0^q \frac{dq}{q-C\varepsilon} = -\frac{1}{RC}\int_0^t dt \implies \ln\left(\frac{q-C\varepsilon}{-C\varepsilon}\right) = -\frac{t}{RC}.
\]
De oplossing moet voldoen aan dat voor $t \to \infty$ de condensator gelijk is aan $Q = C\varepsilon$.
Dan is de tijdsafhankelijkheid van de lading op de condensator
\[
    q(t) = C\varepsilon\left(1-e^{-\frac{t}{RC}}\right) = Q\left(1-e^{-\frac{t}{RC}}\right).
\]
Afleiden naar de tijd levert
\[
    I(t) = \frac{\varepsilon}{R}e^{-\frac{t}{RC}} = I(t=0)e^{-\frac{t}{RC}}.
\]
We definiëren de \textbf{tijdsconstante} $\tau$ voor de RC-kring als
\[
    \tau = RC.
\]
$\tau$ komt overeen met de tijd die nodig is om de lading te verhogen van nul tot 63.2\% van de maximale lading, dus tot
\[
    Q = C\varepsilon(1-\frac{1}{e}).
\]
Gedurende deze tijd is de stroom gereduceerd met de factor $\frac{1}{e}$, dit is tot 36.8\% van zijn initiële waarde.

De hoeveelheid energie die de emk-bron heeft geleverd tijdens het opladen is gelijk aan
\[
    U_{\text{emk}} = \int_0^{U_{\text{emk}}} dU = \int_0^Q \varepsilon\,dq = \varepsilon\int_0^Q dq = Q\varepsilon = C\varepsilon^2.
\]
Uit~\cite[H24]{giancoli} kennen we de opgeslagen energie in de condensator:
\[
    U_C = \frac{C\varepsilon^2}{2}.
\]
Dit is omdat er ook energie werd omgezet in de weerstand:
\begin{align*}
    \int_0^{U_R}dU &= \int_0^\infty P\,dt = \int_0^\infty I^2R\,dt = \int_0^\circ \frac{\varepsilon^2}{R^2}Re^{-\frac{2t}{RC}}\,dt \\
                   &= \frac{\varepsilon^2}{R}\int_0^\infty e^{-\frac{2t}{RC}}\,dt = \frac{\varepsilon^2}{R}\frac{RC}{2}\left(\left.-e^{-\frac{2t}{RC}}\right\vert_0^\infty\right) \implies U_R = \frac{C\varepsilon^2}{2}
\end{align*}
De energie geleverd door de emk werd gelijk verdeeld over de weerstand en de condensator:
\[
    U_{\text{emk}} = U_C + U_R.
\]


\subsection{Ontladen van een RC-keten}

Het ontladen gebeurt met dezelfde $\tau = RC$ als bij het opladen.
Het toepassen van \Cref{stel:kirchhoff2} voor een gesloten lus leidt tot
\begin{align*}
    &\frac{q(t)}{C} - I(t)R = 0 \quad \text{met} \quad I(t) = -\frac{dq(t)}{dt} \\
    &\implies -R \frac{dq(t)}{dt} = \frac{q(t)}{C} \implies -\frac{dq(t)}{q(t)} = \frac{dt}{RC}.
\end{align*}

Integreer tussen $t=0$ en $t$.
In dit tijdsinterval vermindert de lading van $Q$ tot $q(t)$:
\[
    \int_Q^{q(t)}\frac{dq}{q} = -\int_0^t \frac{dt}{RC} \implies \ln\left(\frac{q}{Q}\right) = -\frac{t}{RC}.
\]
Voor de tijdsafhankelijheid van de lading krijgen we
\[
    q(t) = Qe^{-\frac{t}{RC}}.
\]
Dit levert voor de tijdsafhankelijkheid van de stroom
\[
    I(t) = -\frac{dq(t)}{dt} = \frac{Q}{RC}e^{-\frac{t}{RC}} = I(t=0)e^{-\frac{t}{RC}}.
\]
Bij $t=\tau=RC$ is de lading gereduceerd tot 0.368$ Q_{\text{max}}$, de condensator heeft dan 63.2\% van zijn initële lading verloren.
De energiebalans van de RC-keten geeft dan
\begin{align*}
    \int_0^{U_R}dU = \int_0^\infty P\,dt &= \int_0^\infty I^2R\, dt = \frac{Q^2R}{R^2C^2}\int_0^\infty e^{-\frac{2t}{RC}}\,dt = \frac{Q^2}{2C} \\
                   &\implies U_R = \frac{Q^2}{2C}.
\end{align*}
De energie die opgeslagen was in de condensator wordt bij het ontladen door de weerstand omgezet in Joule'se warmte.
