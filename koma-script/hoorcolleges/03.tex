\chapter{Relaties}
\begin{definitie}{Relatie}{relatie}
    Een \textbf{relatie} is een drietal
    \[
        (R,X,Y),
    \] waarin $X$ en $Y$ verzamelingen zijn en $R$ een deelverzameling is van het Cartesisch product $X \times Y$.
    Dus
    \[
        R \subset X \times Y.
    \]
    Meestal zijn de verzamelingen $X$ en $Y$ duidelijk uit de context en dan spreken we van een relatie $R$ van de verzameling $X$ naar $Y$ i.p.v.\ $(R,X,Y)$.

    Een relatie is een eigenschap die kan gelden tussen de elementen van $X$ en de elementen van $Y$.
    De eigenschap geldt tussen $x \in X$ en $y \in Y$ als en slechts als $(x,y) \in R$ (andere notatie is $xRy$).
\end{definitie}
\begin{opmerking}{}
    Een relatie $R$ van $X$ naar $X$ noemen we de relatie $R$ op $X$.
\end{opmerking}
\begin{definitie}{Domein en beeld}{relatie-domein-beeld}
    Zij $R$ een relatie van $X$ naar $Y$.
    Het \textbf{domein} van $R$ is
    \[
        \dom R = \{x \in X \mid \exists y \in Y: (x,y) \in R\}
    \]
    en het \textbf{beeld} van $R$ is
    \[
        \bld R = \{y \in Y \mid \exists x \in X: (x,y) \in R\}.
    \]
\end{definitie}
\begin{definitie}{Eenheidsrelatie}{eenheidsrelatie}
    De \textbf{eenheidsrelatie} op $X$ is
    \[
        I_X = \{(x,x) \in X \times X \mid x \in X\}.
    \]
\end{definitie}
\begin{eigenschap}{}{relaties-gelijkheid}
    Twee relaties $R_1$, van $X_1$ naar $Y_1$, en $R_2$, van $X_2$ naar $Y_2$, zijn gelijk als en slechts als
    \begin{itemize}
        \item $X_1 = X_2$ en $Y_1 = Y_2$ en
        \item $(x,y) \in R_1$ als en slechts als $(x,y) \in R_2$ voor alle $x \in X_1 = X_2$ en voor alle $y \in Y_1 = Y_2$.
    \end{itemize}
\end{eigenschap}
\begin{definitie}{Inverse relatie}{relaties-inverse}
    Zij $R$ een relatie van $X$ naar $Y$, dan is de \textbf{inverse relatie} van $R$ de relatie $R^{-1}$ van $Y$ naar $X$ gedefinieerd door
    \[
        R^{-1} = \{(y,x) \in Y \times X \mid (x,y) \in R\}.
    \]
\end{definitie}
\begin{definitie}{Samengestelde relaties}{relaties-samenstelling}
    Zij $R$ een relatie van $X$ naar $Y$ en $S$ een relatie van $Y$ naar $Z$.
    Dan is de \textbf{samenstelling} $S \circ R$ de relatie van $X$ naar $Z$ gedefinieerd door
    \[
        S \circ R = \{(x,z) \in X \times Z \mid \exists y \in Y: (x,y) \in R\,\, \text{en}\,\, (y,z) \in S\}.
    \]
\end{definitie}


\section{Equivalentierelaties}
\begin{definitie}{}{relaties-drie-belangrijke}
    Zij $R$ een relatie op $X$.
    \begin{enumerate}[label = (\alph*)]
        \item We noemen $R$ \textbf{reflexief} als
            \[
                \forall x \in X: (x,x) \in R.
            \]
        \item We noemen $R$ \textbf{symmetrisch} als
            \[
                \forall x,y \in X: (x,y) \in R \implies (y,x) \in R.
            \]
        \item We noemen $R$ \textbf{transitief} als
            \[
                \forall x,y,z \in X: ((x,y) \in R\,\, \text{en}\,\, (y,z) \in R) \implies (x,z) \in R.
            \]
    \end{enumerate}
\end{definitie}
\begin{definitie}{}{relaties-equivalentierelatie}
    Als een relatie $R$ op $X$ reflexief, symmetrisch en transitief is, dan noemen we $R$ een \textbf{equivalentierelatie} op $X$.

    Als $R$ een equivalentierelatie is en $(x,y) \in R$, dan zeggen we dat $x$ en $y$ \textbf{equivalent} zijn (t.o.v.\ $R$) en we noteren $x \sim_R y$ of kortweg $x \sim y$.
\end{definitie}


\subsection{Partities en equivalentieklassen}
\begin{definitie}{Partitie}{relaties-partitie}
    Een \textbf{partitie $\mathcal{P}$} van een verzameling $X$ is een deelverzameling van de machtsverzameling $P(X)$ met de eigenschappen
    \begin{enumerate}[label = (\alph*)]
        \item De verzamelingen in $\mathcal{P}$ zijn niet leeg, dus
            \[
                \forall A \in \mathcal{P}: A \neq 0.
            \]
        \item De verzamelingen in $\mathcal{P}$ zijn onderling disjunct, dus
            \[
                \forall A, B \in \mathcal{P}: A \neq B \implies A \cap B \neq \emptyset.
            \]
        \item De verzamelingen in $\mathcal{P}$ overdekken $X$, dus
            \[
                \forall x \in X: \exists A \in \mathcal{P}: x \in A.
            \]
            Dit is equivalent met
            \[
                X = \bigcup_{A \in \mathcal{P}} A.
            \]
    \end{enumerate}
\end{definitie}
\begin{definitie}{Equivalentieklasse}{relaties-equivalentieklasse}
    Zij $\sim$ een equivalentierelatie op $X$ en zij $x \in X$.
    Dan is $\{y \in X \mid x \sim y\}$ de \textbf{equivalentieklasse} van $x$ m.b.t.\ $\sim$.
    Dit wordt genoteerd met $\lbrack x \rbrack_\sim$ of kortweg $\lbrack x \rbrack$ als het duidelijk is welke equivalentierelatie bedoeld wordt.
    Dus
    \[
        \lbrack x \rbrack = \{y \in X \mid x \sim y\}.
    \]
    De verzameling van alle equivalentieklassen van $\sim$ wordt genoteerd met $X \slash \sim$.
    Dus
    \begin{align*}
        X \slash \sim &= \{\lbrack x \rbrack \mid x \in X\} \\
                      &= \{A \subset X \mid \exists x \in X: A = \lbrack x \rbrack\}.
    \end{align*}
    De verzameling $X \slash \sim$ wordt de \textbf{quotiëntverzameling} van $X$ t.o.v.\ de equivalentierelatie $\sim$ genoemd.
\end{definitie}
Dit leidt tot de belangrijkste stelling over equivalentierelaties.

\begin{stelling}{}{relaties-belangrijk}
    Zij $\sim$ een equivalentierelatie op $X$.
    Dan is $X \slash \sim$ een partitie van $X$.
\end{stelling}
Om dit te bewijzen hebben we eenvoudigere hulpresultaten nodig.
Zulke hulpresultaten worden vaak \textbf{lemma's} genoemd.

\begin{lemma}{}{relaties-belangrijk}
    Zij $\sim$ een equivalentierelatie op $X$.
    Dan geldt voor alle $x,y \in X$:
    \begin{enumerate}[label = (\alph*)]
        \item $x \in \lbrack x \rbrack \,\, \text{en}\,\, \lbrack x \rbrack = \emptyset$,
        \item als $x \sim y$, dan $\lbrack x \rbrack = \lbrack y \rbrack$,
        \item als $\neg(x \sim y)$, dan $\lbrack x \rbrack \cap \lbrack y \rbrack = \emptyset$.
    \end{enumerate}
\end{lemma}
\begin{bewijs}{}{}
    \begin{enumerate}[label = (\alph*)]
        \item Omdat $\sim$ reflexief is, geldt $x \in \lbrack x \rbrack$ en dus is $\lbrack x \rbrack$ niet leeg.
        \item Neem aan dat $x \sim y$.
            We willen bewijzen dat $\lbrack x \rbrack = \lbrack y \rbrack$ en hiertoe bewijzen we twee inclusies.
            \begin{itemize}
                \item Bewijs dat $\lbrack x \rbrack \subset \lbrack y \rbrack$: Kies $z \in \lbrack x \rbrack$ willekeurig.
                    Dit betekent dat $x \sim z$.
                    We weten ook dat $x \sim y$ en vanwege symmetrie is dan ook $y \sim x$.
                    Uit $y \sim x$ en $x \sim z$ volgt, vanwege transitiviteit, dat $y \sim z$ en dit betekent dat $z \in \lbrack y \rbrack$.
                    We hebben $z \in \lbrack x \rbrack$ willekeurig gekozen en dus volgt er dat $\lbrack x \rbrack \subset \lbrack y \rbrack$.
                \item Bewijs dat $\lbrack y \rbrack \subset \lbrack x \rbrack$: Kies $z \in \lbrack y \rbrack$ willekeurig.
                    Dan geldt $y \sim z$.
                    Er geldt ook $x \sim y$ en vanwege transitiviteit is dan ook $x \sim z$.
                    Dit betekent $z \in \lbrack x \rbrack$.
                    Omdat $z \in \lbrack y \rbrack$ willekeurig gekozen is volgt er dat $\lbrack y \rbrack \subset \lbrack x \rbrack$.
            \end{itemize}
            Beide inclusies zijn bewezen dus er geldt inderdaad $\lbrack x \rbrack = \lbrack y \rbrack$.
        \item We moeten bewijzen dat $\neg(x \sim y) \implies \lbrack x \rbrack \cap \lbrack y \rbrack = \emptyset$.
            We doen dit m.b.v.\ contrapositie.
            We bewijzen dus
            \[
                \lbrack x \rbrack \cap \lbrack y \rbrack \neq \emptyset \implies x \sim y.
            \]
            Neem aan dat $\lbrack x \rbrack \cap \lbrack y \rbrack \neq \emptyset$.
            Dan is er een $z \in \lbrack x \rbrack \cap \lbrack y \rbrack$.
            Omdat $z \in \lbrack x \rbrack$ geldt $x \sim z$ en omdat $z \in \lbrack y \rbrack$ geldt $y \sim z$.
            Vanwege de symmetrie is dan ook  $z \sim y$.
            Uit $x \sim z$ en $z \sim y$ volgt dat $x \sim y$ vanwege de transitiviteit van de equivalentierelatie.
    \end{enumerate}
            Dit is wat we moesten bewijzen.
\end{bewijs}
Nu kunnen we \Cref{stel:relaties-belangrijk} bewijzen.
\begin{bewijs}{\Cref{stel:relaties-belangrijk}}{}
    We moeten bewijzen dat $X \slash \sim$ voldoet aan de drie voorwaarden van \Cref{def:relaties-partitie}.
    \begin{enumerate}[label = (\alph*)]
        \item Als $A \in X \slash \sim$, dan is er een $x \in X$ met $A = \lbrack x \rbrack$.
            Op grond van onderdeel (a) van het lemma is $A \neq \emptyset$.
        \item Neem aan dat $A, B \in X \slash \sim$ twee elementen van $X \slash \sim$ zijn.
            Er zijn $x$ en $y$ in $X$ met $A = \lbrack x \rbrack$ en $B = \lbrack y \rbrack$.
            Er zijn twee mogelijkheden, nl.\ $x \sim y$ en $\neg(x \sim y)$.
            Als $x \sim y$ dan volgt uit onderdeel (b) van \Cref{lem:relaties-belangrijk} dat $A = B$ en als $\neg(A \sim B)$, dan volgt uit onderdeel (c) van het lemma dat $A \cap B = \emptyset$.

            Als $A \neq B$ volgt dus dat $A \cap B = \emptyset$, hetgeen betekent dat de elementen van $X \slash \sim$ onderling disjunct zijn.
        \item Kies $x \in X$.
            Dan is $x \in \lbrack x \rbrack$ vanwege onderdeel (a) van \Cref{lem:relaties-belangrijk}.
            Er is dus een $A \in X \slash \sim$ met $x \in A$ (nl.\ $A = \lbrack x\rbrack$).
    \end{enumerate}
\end{bewijs}

\begin{stelling}{}{}
    Zij $\mathcal{P}$ een partitie van $X$.
    Dan definieert
    \[
        x \sim y \iff (\exists A \in \mathcal{P}: x \in A \land y \in A)
    \] een equivalentierelatie op $X$.
    De equivalentieklassen van $\sim$ zijn precies de elementen van $\mathcal{P}$.
    Dus
    \[
        X \slash \sim = \mathcal{P}.
    \]
\end{stelling}
\begin{bewijs}{}{}
    Dit is een oefening.
\end{bewijs}
