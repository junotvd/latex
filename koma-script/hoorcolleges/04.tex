\chapter{Functies}
We zien een functie hier als een relatie met een speciale eigenschap.
\begin{definitie}{}{functie}
    Zij $f$ een relatie van een verzameling $X$ naar een verzameling $Y$.
    We noemen $f$ een \textbf{functie} van $X$ naar $Y$ als voor elke $x \in X$ er precies één $y \in Y$ bestaat met $(x,y) \in f$.
\end{definitie}
\begin{opmerking}{}
    Je kan een functie dus ook definiëren als $(f,X,Y)$ om nadruk te leggen op de verzameling, $X \times Y$, waarvan $f$ een deelverzameling is met de eigenschap dat er voor elke $x \in X$ precies één $y \in Y$ is met $(x,y) \in f$.
\end{opmerking}
\begin{voorbeeld}{}{}
    De eenheidsrelatie op een verzameling $X$ is gegeven door
    \[
        I_X = \{(x,x) \in X \times X \mid x \in X\}.
    \]
    Dit is een functie van $X$ naar $X$ die ook de \textbf{eenheidsfunctie} wordt genoemd.
    Voor deze functie geldt $I_X(x) = x$ voor elke $x \in X$.
\end{voorbeeld}

Uit \Cref{def:functie} volgt dat $y$ uniek bepaald wordt door $x$.
We zeggen dat $y$ het \textbf{beeld} is van $x$ onder $f$.
Een uitgebreide notatie voor $f$ is
\[
    f: X \to Y: x \mapsto f(x).
\]

Vaak specificeren we een functie $f: X \to Y$ door het beeld $f(x)$ te definiëren.
Dan is $f(x)$ het \textbf{functievoorschrift} voor de functie.
Dit moet gedefinieerd zijn voor elke $x \in X$.

Het is belangrijk om een verschil te maken tussen de functie $f$ en het functievoorschrift $f(x)$.
\begin{voorbeeld}{}{}
    We spreken over de sinus-functie $\sin$ met als functievoorschrift $\sin x$.
    Het is niet de functie $\sin x$.
    In plaats daarvan kunnen we zeggen dat we een functie $f: \mathbb{R} \to \mathbb{R}$ beschouwen met $f(x) = \sin x$ voor elke $x \in \mathbb{R}$.
    Hiermee bedoelen we dan dat
    \[
        f = \{(x,y) \in \mathbb{R} \times \mathbb{R} \mid y = \sin x\}.
    \]
\end{voorbeeld}
\begin{definitie}{Domein}{functies-domein}
    Zij $f$ een functie van $X$ naar $Y$.
    Dan is $X$ het \textbf{domein} van $f$ en $Y$ het \textbf{codomein} van $f$.
\end{definitie}

\begin{stelling}{}{functies-gelijkheid}
    Zij $f: X_1 \to Y_1$ en $g: X_2 \to Y_2$ twee functies.
    Dan is $f = g$ als en slechts als $X_1 = X_2$ en $Y_1 = Y_2$ en
    \[
        \forall x \in X_1: f(x) = g(x).
    \]
\end{stelling}

Het is ook mogelijk om functies samen te stellen, dit is gekend vanuit het middelbaar.
\begin{definitie}{}{functies-samenstellen}
    Neem aan dat $f$ een functie is van $X$ naar $Y$ en dat $A \subset X$.
    Dan kunnen we een functie $g: A \to Y$ definiëren door
    \[
        g(x) = f(x)
        \quad \text{voor elke}\; x \in A.
    \]
    Deze functie is de \textbf{beperking} van $f$ tot $A$ en wordt genoteerd door $f\vert_A$.
\end{definitie}
\begin{stelling}{}{functies-samenstellen}
    Neem aan dat $f: X \to Y$ en $g: Y \to Z$.
    Dan is $g \circ f$ een functie van $X$ naar $Z$ en er geldt
    \begin{equation}
        (g \circ f)(x) = g(f(x)) \quad \text{voor elke}\; x \in X.
        \label{eq:functies-samenstellen}
    \end{equation}
\end{stelling}
\begin{bewijs}{}{}
    In dit bewijs gebruiken we de notatie m.b.v.\ verzamelingen om functies en relaties weer te geven.
    Volgens \Cref{def:relaties-samenstelling} is
    \begin{equation}
        g \circ f = \{(x,z) \in X \times Z \mid \exists y \in Y: (x,y) \in f \; \text{en}\; (y,z) \in g\}.
        \label{eq:stel-samenstellen}
    \end{equation}
    Neem $x \in X$ willekeurig.
    We moeten laten zien dat er precies één $z \in Z$ is met $(x,z) \in g \circ f$.

    Eerst bewijzen we de \emph{existentie} van $z$.
    Neem $y = f(x)$ en $z = g(y)$.
    Dan is $(x,y) \in f$ en $(y,z) \in g$, zodat uit \cref{eq:stel-samenstellen} volgt dat $(x,z) \in g \circ f$.
    Er is dus zeker een $z \in Z$ met $(x,z) \in g \circ f$.

    Vervolgens bewijzen we de \emph{uniciteit} van $z$.
    Dit doen we door aan te nemen dat $z_1, z_2 \in Z$ voldoen aan $(x, z_1) \in g \circ f$ en $(x,z_2) \in g \circ f$ en daaruit te bewijzen dat $z_1 = z_2$.
    Neem dus aan dat $z_1, z_2 \in Z$ met $(x, z_1) \in g \circ f$ en $(x,z_2) \in g \circ f$.
    Dan volgt uit \cref{eq:stel-samenstellen} dat er een $y_1 \in Y$ bestaat met $(x,y_1) \in f$ en $(y_1, z_1) \in g$.
    Net zo is er een $y_2 \in Y$ met $(x,y_2) \in f$ en $(y_2, z_2) \in g$.
    Omdat $f$ een functie is en $(x,y_1) \in f$ en $(x,y_2) \in f$, volgt $y_1 = y_2$.
    Dan geldt $(y_1,z_2) = (y_2,z_2) \in g$.
    Omdat ook $(y_1,z_1) \in g$ en $g$ een functie is, volgt hieruit dat $z_1 = z_2$.
    Dit bewijst de uniciteit.

    We hebben nu bewezen dat $f \circ g$ een functie is van $X$ naar $Z$.
    In het bovenstaande existentie-bewijs is al laten zien dat als $y = f(x)$ en $z = g(x)$ dat dan $(x,z) \in (g \circ f)$.
    Omdat  $g \circ f$ een functie is, kunnen we schrijven $z = (g \circ f)(x)$.
    Ook geldt $z = g(y) = g(f(x))$.
    Dit bewijst \cref{eq:functies-samenstellen}.
\end{bewijs}

\begin{eigenschap}{}{functies-samengesteld-associatief}
    Het samenstellen van functies is \textbf{associatief}.
\end{eigenschap}


\subsection{Beeld en invers beeld}
\begin{definitie}{}{functies-invers-beeld}
    Zij $f: X \to Y$ een functie.
    Zij $A$ een deelverzameling van $X$.
    Dan is $f(a)$ de deelverzameling van $Y$ gegeven door
    \[
        f(A) = \{y \in Y \mid \exists x \in A: f(x) = y\}.
    \]
    We noemen $f(A)$ het \textbf{beeld} van $A$.

    Zij $B$ een deelverzameling van $Y$.
    Dan is $f^{-1}(B)$ de deelverzameling van $X$ gegeven door
    \[
        f^{-1}(B) = \{x \in X \mid f(x) \in B\}.
    \]
    We noemen $f^{-1}(B)$ het \textbf{invers beeld} van $B$.
\end{definitie}


\section{Injecties, surjecties en bijecties}
\begin{definitie}{}{injectie}
    Een functie $f: X \to Y$ is \textbf{injectief} (of één op één) als ze verschillende waarden aanneemt in verschillende elementen van het domein, dus
    \[
        \forall x_1, x_2 \in X: x_1 \neq x_2 \implies f(x_1) \neq f(x_2).
    \]
    In dat geval noemen we $f$ een injectieve functie of een \textbf{injectie}.
\end{definitie}
Met behulp van contrapositie kunnen we de injectiviteit ook uitdrukken als
\[
    \forall x_1, x_2 \in X: f(x_1) = f(x_2) \implies x_1 = x_2.
\]
Een functie $f$ is injectief als en slechts als voor elke $y \in Y$ het inverse beeld $f^{-1}(y)$ uit hoogstens één element bestaat.

\begin{definitie}{}{surjectie}
    Een functie $f: X \to Y$ is \textbf{surjectief} as elk element van het codomein als beeld optreedt; dus als
    \[
        \forall y \in Y: \exists x \in X: f(x) = y.
    \]
    In dat geval noemen we $f$ een surjectieve functie of een \textbf{surjectie}.
\end{definitie}
Een functie $f$ is surjectief als en slechts als voor elke $y \in Y$ het inverse beeld $f^{-1}(y)$ uit minstens één element bestaat (dus niet-leeg is).

\begin{definitie}{}{bijectie}
    Een functie is \textbf{bijectief} als ze zowel injectief als surjectief is.
    We noemen zo een functie een \textbf{bijectie}.
\end{definitie}
Een functie $f$ is bijectief als en slechts als voor elke $y \in Y$ het inverse beeld $f^{-1}(y)$ uit precies één element bestaat.


\subsection{Inverse functies}
\begin{definitie}{}{inverteerbaarheid}
    Een functie $f: X \to Y$ is \textbf{inverteerbaar} als er een functie $g: Y \to X$ bestaat met de eigenschappen dat
    \[
        g \circ f = I_X
        \quad \text{en} \quad
        f \circ g = I_Y.
    \]
    In dat geval noemen we $g$ een \textbf{inverse functie} van $f$.
\end{definitie}

\begin{stelling}{}{inverteerbaarheid}
    Zij $f: X \to Y$ een functie.
    Dan is $f$ inverteerbaar als en slechts als $f$ bijectief is.
    Bovendien is voor een inverteerbare functie de inverse functie uniek bepaald.
\end{stelling}
\begin{bewijs}{}{}
    Zie~\cite[p. 61]{cursus}.
\end{bewijs}

Een injectieve functie die niet bijectief is kan bijectief gemaakt worden door het codomein te beperken. Zij namelijk $f: X \to Y$ injectief.
Dan is de functie
\[
    \hat{f}: X \to f(X): x \mapsto \hat{f}(x) = f(x)
\] die we uit $f$ verkrijgen door het codomein van $Y$ te beperken tot $f(X)$, zowel injectief als surjectief.
Dus is $\hat{f}$ bijectief en bijgevolg inverteerbaar vanwege \Cref{stel:inverteerbaarheid}.
De inverse functie ${(\hat{f})}^{-1}$ bestaat dus.
