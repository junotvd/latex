% Hoofdstuk 6 wordt dit jaar niet behandeld.
\setcounter{chapter}{6}


\chapter{Orderelaties}
\les{Hoorcollege}{7}{wo 6 nov 2024 14:00}{Orderelaties}
Wekelijkse taken vanaf week 8 per twee, in \LaTeX.

\begin{definitie}{}{anti-symmetrisch}
    Een relatie $R$ op een verzameling  $X$ is \textbf{anti-symmetrisch} als
    \[
        \forall x,y \in X: \lbrack(x,y) \in R \land (y,x) \in R\rbrack \implies x = y.
    \]
    Equivalent hiermee is
    \[
        \forall x,y \in X \lbrack(x,y) \in R \land x \neq y\rbrack \implies (y,x) \notin R.
    \]
\end{definitie}

\begin{definitie}{}{orderelatie}
    Een relatie $X$ is een \textbf{orderelatie} op $X$ als ze reflexief, anti-symmetrisch en transitief is.
    Een orderelatie op $X$ wordt ook wel een \textbf{partiële ordening} op $X$ genoemd.
\end{definitie}

\begin{voorbeeld}{}{orderelaties}
    \begin{enumerate}[label = (\alph*)]
        \item De relatie, $K$, op $\mathbb{R}$ (`kleiner dan of gelijk aan')
            \[
                K = \{(x,y) \in \mathbb{R} \times \mathbb{R} \mid x \leq y \}
            \]
            is een orderelatie op $\mathbb{R}$.\label{item:orderelatie-kdg}
        \item Zij $X$ een verzameling, dan is
            \[
            \{(A,B) \in P(X)\times P(X) \mid A \subset B \}
        \] een orderelatie op de machtsverzameling van $X$.\label{item:orderelatie-P}
        \item De relatie
            \[
                D = \{(n,m) \in \mathbb{N}_0 \times \mathbb{N}_0 \mid n\; \text{is een deler van}\; m\}
            \] is een orderelatie op de strikt positieve gehele getallen.\label{item:orderelatie-gehele-getallen}
    \end{enumerate}
\end{voorbeeld}

Je ziet dat bij een orderelatie $R$ dat $(x,y) \in R$ betekent dat $x$ in een of andere zin kleiner dan of gelijk is aan $y$.
Concreet geven we dit weer met $\leq$ of $\subset$ maar in een algemene context zullen we $\preceq$ gebruiken om een orderelatie aan te duiden.

\begin{definitie}{}{}
    Een \textbf{partieel geordende verzameling} is een koppel $(X, \preceq)$ bestaande uit een verzameling $X$ en een orderelatie $\preceq$ op $X$.
\end{definitie}
Als $(X, \preceq)$ een geordende verzameling is, dan schrijven we meestal $x \preceq y$ als het koppel $(x,y)$ tot de relatie $\preceq$ behoort.
We gebruiken ook de volgende voor de hand liggende varianten op deze notatie:
\begin{itemize}
    \item $x \succeq y$ betekent $y \preceq x$,
    \item $x \prec y$ betekent $x \preceq$ en $x \neq y$,
    \item $x \succ y$ betekent $x \succeq$ en $x \neq y$.
\end{itemize}

\begin{definitie}{}{}
    Een orderelatie $\preceq$ op $X$ is een \textbf{totale ordening} als
    \[
        \forall x,y \in X: x \preceq y \lor y \preceq x.
    \]
\end{definitie}
In een totaal geordende verzameling $(X, \preceq)$ kunnen we elk tweetal elementen $x$ en $y$ met elkaar vergelijken.
Er geldt telkens precies één van de drie mogelijkheden:
\[
    x \prec y, \quad x = y, \quad x \succ y.
\]
We bekijken de orderelaties uit \Cref{vb:orderelaties}.
$K$ uit \cref{item:orderelatie-kdg} is totaal.
De orderelatie $D$ uit \cref{item:orderelatie-gehele-getallen} is niet totaal.
Tenslotte is de orderelatie uit \cref{item:orderelatie-P} niet totaal als $X$ uit meer dan één element bestaat.
\begin{voorbeeld}{}{}
    \( (P(X), \subset) \) is niet totaal geordend als \( \lvert X\rvert \leq z \).
\end{voorbeeld}

\begin{definitie}{}{orde-extrema}
    Zij $A \subset X$.
    We noemen $a$ een \textbf{grootste element}
    van $A$ als $a \in A$ en
    \[
        \forall x\in A: x \preceq a.
    \]
    Analoog kennen we het \textbf{kleinste element} van $A$.
\end{definitie}

\begin{definitie}{}{orde-sup-inf}
    Zij $(X, \preceq)$ een geordende verzameling en $A \subset X$.
    \begin{enumerate}[label = (\alph*)]
        \item Dan is $b \in X$ een \textbf{bovengrens} van $A$ als
            \[
                \forall x \in A: x \preceq b.
            \]
            Als $A$ een bovengrens heeft, dan noemen we $A$ \textbf{naar boven begrensd}.
        \item Verder is $s \in X$ een \textbf{supremum} of \textbf{kleinste bovengrens} van $A$ als $s$ een bovengrens is van $A$ die kleiner is dan elke andere bovengrens van $A$.
    \end{enumerate}
    Analoog kennen we \textbf{ondergrens} en \textbf{infimum}.
\end{definitie}
\begin{eigenschap}{}{}
    \begin{enumerate}
        \item Als \(\sup(A)\) bestaat en \(\sup(X) \in A\), dan \(\sup(A) = \max(A)\).
        \item Als \(\max(A)\) bestaat, dan \(\max(A) = \sup(A)\).
    \end{enumerate}
\end{eigenschap}
\begin{voorbeeld}{}{}
    \( (\mathbb{Q}, \leq) \text{ en } (\mathbb{R}, \geq) \) zijn totaal geordende verzamelingen.
\end{voorbeeld}
\begin{definitie}{}{}
    Zij \( (X, \preceq) \) een geordende verzameling.
    De ordening \(\preceq\) noemen we \textbf{volledig} als
    \begin{itemize}
        \item elke niet lege deelverzameling van $X$ die een bovengrens heeft, ook een kleinste bovengrens heeft in $X$.
        \item elke niet lege deelverzameling van $X$ die een ondergrens heeft, ook een grootste ondergrens heeft in $X$.
    \end{itemize}
\end{definitie}
\begin{opmerking}{}{}
    \( (\mathbb{Q}, \leq) \) is \underline{niet} volledig.\ \textit{(Volgende week: \((\mathbb{R}, \leq)\) is \underline{wel} volledig.)}
\end{opmerking}
\begin{stelling}{}{}
    Als $A$ een bovengrens heeft, dan is het supremum de kleinste bovengrens, en bestaat er geen kleinere bovengrens.
\end{stelling}
\noindent We proberen te bewijzen dat er bij elke bovengrens van $A$ een strikt kleinere bovengrens te vinden valt.
\begin{bewijs}{}{}
    Beschouw \(A = \{ x \in \mathbb{Q} \mid x \geq 1 \land x^2 < 2\} \).
    \begin{itemize}
        \item \(I \in A\), dus \(A \neq \emptyset\) \checkmark
        \item \(z\) is een bovengrens \checkmark
    \end{itemize}

    Zij $s$ een bovengrens van $A$, \(s \in \mathbb{R}\).  Dan \(s\geq 1\) en \(s^2 \neq 2\). Dan zijn er nog twee mogelijkheden:
    (\textit{Er is geen rationaal getal $r$ met \(r^2=2\).})

    \begin{itemize}
        \item \(s^2<2\):

            We laten zien dat er een \(n\in\mathbb{N}_0\) met \({(s+\frac{1}{n})}^2<2\) Dan is \(x=s+\frac{1}{n}\) rationaal met \(x^2<2\). Dus \(x\in A \text{ en } x > s\). Dit is een tegenspraak.

            Neem $n$ zo groot dat \(n\in \mathbb{N}_0 > \frac{2s+1}{2-s^2}\). Dan is
            \[
                \frac{1}{n} < \frac{2-s^2}{2s+1}
            \]
            waardoor
            \[
                s + \frac{1}{n} < \frac{s(2s+1)}{2s+1} + \frac{2-s^2}{2s+1} = \frac{s^2+s+2}{2s+1}
            \]
            De vraag is nu of \({\left(\frac{s^2+s+2}{2s+1}\right)}^2 < 2\). We weten \(s^2<2\).

        \item \(s^2>2\):

            Zoek $n$ groot genoeg met
            \[
                {\left(s-\frac{1}{n}\right)}^2 > 2 \iff\underbrace{\dotsc}_{\textit{klad}} \iff \frac{2s}{s^2-2}<2
            \]
    \end{itemize}
\end{bewijs}


\section*{Informatie over de \LaTeX~opdracht}
Er worden punten afgetrokken voor spelling, zinsbouw, grammatica\ldots Let op wetenschappelijke schrijfstijl. Het moet een `verhaal' zijn. Vermijd pijlen.
