\documentclass[
    10pt,
    a4paper,
    twoside,
]{article}


% \usepackage[T1]{fontenc}
% \usepackage{PTSerif}
\usepackage{fontspec}
% \usepackage[bitstream-charter]{mathdesign}

\usepackage[main=dutch,provide=*]{babel}


% \usepackage{geometry}
% \geometry{%
%     paperwidth=210mm,
% 	paperheight=297mm,
%     top=72pt,
%     marginparsep=20pt,
%     marginparwidth=60pt,
%     head=24pt,
%     textheight=690pt,
%     footskip=40pt,
% }


\usepackage{float}
\usepackage{ragged2e}
\usepackage{needspace}
\usepackage{xspace}

\usepackage{csquotes}
\usepackage{parskip}

\usepackage{tikz}
\usepackage{graphicx}

\usepackage{booktabs}
\usepackage{tabularx}
\usepackage{enumitem}

\usepackage{amsmath, amsfonts, mathtools, amsthm, amssymb}
\usepackage{bm, cancel}

\usepackage{microtype} %load after amsthm

% \usepackage{showframe}
% \usepackage{layout}
\usepackage{lipsum}

\usepackage[
    sorting = nyt,
    style = alphabetic
]{biblatex}

\usepackage[
    inner=30mm,
    outer=30mm,
    top=25mm,
    bottom=25mm
]{geometry}


\usepackage[
    colorlinks=true,
    linkcolor=black,
    filecolor=magenta,
    citecolor=black,
    urlcolor=cyan,
    pdfpagemode=FullScreen,
]{hyperref}


\usepackage{fancyhdr}
\fancypagestyle{plain}{%
    \fancyhf{}
    \fancyfoot[R]{\thepage}
}

\pagestyle{fancy}
\renewcommand{\headrulewidth}{0pt}

\fancyhead[LE, RO]{\thepage}
\fancyhead[RE]{\slshape\nouppercase{\leftmark}}
\fancyhead[LO]{\slshape\nouppercase{\rightmark}}
\fancyfoot[C]{}


\usepackage[dutch]{cleveref}
\crefname{equation}{}{}
\Crefname{equation}{Vergelijking}{Vergelijkingen}


\renewcommand\geq{\geqslant}
\renewcommand\leq{\leqslant}

\let\epsilon\varepsilon
% \let\phi\varphi

\DeclareMathOperator{\dom}{dom}
\DeclareMathOperator{\bld}{bld}


\usepackage{xifthen}
\makeatletter
\def\@les{}%
\newcommand{\les}[3]{%
    \def\@les{#1 #2}%
    {\def\@les{#1 #2}}
    \subsection*{\@les}
    \marginpar{\small\textsf{\mbox{#3}}}
}
\makeatother

\usepackage{import}
\usepackage{pdfpages}
\usepackage{transparent}
\newcommand{\incfig}[2][1]{%
    \def\svgwidth{#1\columnwidth}
    \import{./figures/}{#2.pdf_tex}
}
% \pdfsuppreswarningpagegroup=1


\usepackage[most]{tcolorbox}

\tcbset{%
    parbox = false,
}
\tcbset{%
    boxstyle/.style = {%
        parbox=false,
        enhanced,
        sharp corners,
        colframe = black,
        coltitle = black,
        colback = white,
        fonttitle = \bfseries,
        after title = {.},
        description font = \mdseries,
        attach boxed title to top left = {xshift=2mm, yshift=-2.5mm},
        separator sign none,
        description delimiters parenthesis,
        boxed title style = {%
            colframe = white,
            colback = white,
            sharp corners,
        },
        boxrule = 0.5pt,
    },
    proofstyle/.style = {%
        parbox=false,
        enhanced,
        blanker,
        breakable,
        coltitle = black,
        fonttitle = \bfseries,
        description font = \mdseries,
        attach title to upper = {.\;},
        left = 5mm,
        top = 1mm,
        borderline west = {0.5pt}{0pt}{black},
        description delimiters parenthesis,
        separator sign none,
    },
    plainstyle/.style = {%
        parbox=false,
        enhanced,
        sharp corners,
        boxrule = 0pt,
        boxsep = 0mm,
        left = 0mm,
        breakable,
        attach title to upper = {.\:},
        % after title = {\;},
        description font = \mdseries,
        coltitle = black,
        colback = white,
        colframe = white,
        fonttitle = \itshape,
        description delimiters parenthesis,
        separator sign none,
    },
    plainstylebold/.style = {%
        parbox=false,
        enhanced,
        sharp corners,
        boxrule = 0pt,
        boxsep = 0mm,
        left = 0mm,
        breakable,
        attach title to upper = {.\:},
        % after title = {\;},
        description font = \mdseries,
        coltitle = black,
        colback = white,
        colframe = white,
        fonttitle = \bfseries,
        description delimiters parenthesis,
        separator sign none,
    },
    wrongstyle/.style = {%
        colback = red,
    },
}


\newtcbtheorem[auto counter,number within=section,crefname={definitie}{definities}]%
    {definitie}{Definitie}{boxstyle}{def}

\newtcbtheorem[use counter from=definitie,crefname={axioma}{axiomas}]%
    {axioma}{Axioma}{boxstyle}{axi}

\newtcbtheorem[use counter from=definitie,crefname={eigenschap}{eigenschappen}]%
    {eigenschap}{Eigenschap}{plainstylebold,breakable}{eig}

\newtcbtheorem[use counter from=definitie,crefname={lemma}{lemmas}]%
    {lemma}{Lemma}{boxstyle,fontupper=\itshape}{lem}

\newtcbtheorem[use counter from=definitie,crefname={stelling}{stellingen}]%
    {stelling}{Stelling}{boxstyle,fontupper=\itshape}{stel}

\newtcbtheorem[use counter from=definitie]%
    {bewering}{Bewering}{boxstyle,fontupper=\itshape}{bewr}

\newtcbtheorem[use counter from=definitie,crefname={propositie}{proposities}]%
    {propositie}{Propositie}{boxstyle,fontupper=\itshape}{prop}

\newtcbtheorem[use counter from=definitie,crefname={gevolg}{gevolgen}]%
    {gevolg}{Gevolg}{boxstyle,breakable}{gev}

\newtcbtheorem[no counter,crefname={bewijs}{bewijzen}]%
    {bewijs}{Bewijs}{plainstyle,breakable,top=0mm}{bew}

\newtcbtheorem[use counter from=definitie,crefname={voorbeeld}{voorbeelden}]%
    {voorbeeld}{Voorbeeld}{plainstylebold,breakable}{vb}

\newtcbtheorem[use counter from=definitie,crefname={toepassing}{toepassingen}]%
    {toepassing}{Toepassing}{plainstylebold,breakable}{tp}

\newtcbtheorem[auto counter,number within=section,crefname={oefening}{oefeningen}]%
    {oefening}{Oefening}{plainstylebold,breakable}{oef}

\AtEndEnvironment{bewijs}{\qed}
\AtEndEnvironment{voorbeeld}{\null\hfill$\diamond$}


\newtcolorbox{oplossing}{plainstyle,title=Oplossing,top=0mm}
\newtcolorbox{antwoord}{plainstyle,title=Antwoord,top=0mm}
\newtcolorbox{uitwerking}{plainstyle,title=Uitwerking,top=0mm}
\newtcolorbox{opmerking}{plainstyle,title=Opmerking}
\newtcolorbox{hint}{plainstyle,title=Hint,top=0mm,fonttitle=\itshape,fontupper=\itshape,top=0mm}
\newtcolorbox{herinner}{plainstyle,title={Herinner}}
\newtcolorbox{intermezzo}{plainstyle,title=Intermezzo}
\newtcolorbox{notatie}{plainstyle,title=Notatie}
\newtcolorbox{conventie}{plainstyle,title=Conventie}
\newtcolorbox{interpretatie}{plainstyle,title=Interpretatie}
\newtcolorbox[auto counter]{vraag}{plainstyle,title={Vraag~\thetcbcounter}}



\author{Junot Van Dijck}

\addbibresource{ref.bib}

\title{Article}


\begin{document}

\maketitle

\section{Meervoudige integralen}

In het boek:~\cite[H15.1 - H15.6]{cursus}

\begin{hint}{}
    Bij dubbele en meervoudige integralen is het nuttig om een tekening te maken van het domein waarover je integreert.
    Dit kan helpen om een geschikt coördinatensysteem te kiezen en de integratiegrenzen te bepalen.
\end{hint}


\subsection*{Dubbele integralen}

\begin{oefening}{}{}
    Bereken de dubbele integraal
    \[
        \iint_S (x-3y)\,dA
    \] waarbij $S$ de driehoek is met hoekpunten $(0,0)$, $(1,0)$ en $(0,c)$ waarbij $c \in \mathbb{R}$.
\end{oefening}

\begin{oefening}{}{}
    Bereken
    \[
        \iint_D (3-2x)\, dA
    \] waarbij $D$ de helft is van de schijf $x^2 + y^2 = 4$ met $y \geq 0$.
    \begin{hint}{}
        Gebruik symmetrie om het rekenwerk te beperken.
    \end{hint}
\end{oefening}


\subsection*{Oneigenlijke integralen}

\begin{oefening}{}{}
    Convergeert de oneigenlijke integraal
    \[
        \iint_K \frac{dA}{(1+x^2)(1+y^2)}
    \] waarbij $K$ het eerste kwadrant in het $xy$-vlak is?
    Zo ja, bepaal dan ook de waarde waarnaar deze integraal convergeert.
\end{oefening}


\subsection*{Transformatie van coördinaten}

\begin{oefening}{}{}
    Beschouw de verzameling $D = \{(x,y) \in \mathbb{R}^2 \mid x \geq 0, y \geq 0, 1 \leq \sqrt{x^2 + y^2} \leq 3\}$.
    Maak een schets van $D$ en bepaal vervolgens de integraal
    \[
        \iint_D e^{x^2+y^2}\, dA.
    \]
\end{oefening}


\subsection*{Drievoudige integralen en transformatie van coördinaten in drie dimensies}
\begin{oefening}{}{}
    Bereken de drievoudige integraal
    \[
        \iiint_R yz^2 e^{-xyz}\, dV
    \] waarbij $R$ de kubus $0 \leq x, y, z \leq 1$ is.
\end{oefening}

\begin{oefening}{}{}
    Bepaal het volume dat ingesloten wordt tussen de kegel $z = \sqrt{x^2 + y^2}$ en de sfeer $x^2 + y^2 + z^2 = a^2$ met $a \in \mathbb{R}$.
    \begin{hint}{}
        In een goed gekozen coördinatensysteem kun je dit vinden met minimaal rekenwerk.
    \end{hint}
\end{oefening}


\section{Riemann-integraal van twee veranderlijken}

Zij $\displaystyle f\colon D \subset \mathbb{R}^2 \to \mathbb{R}\colon (x,y) \mapsto f(x,y)$ begrensd en positief.
We kunnen het volume onder het oppervlak van $f$ benaderen door volumes van balken op te tellen.
\begin{itemize}
    \item Deel $D$ op in gebieden $D_1, D_2, \ldots, D_n$ met $P = \{D_1, \ldots, D_n\}$ een \emph{partitie} van $D$.
    \item Kies $\vec{p}_k \in D_k \subset D$ en we noteren de oppervlakte van $D_k$ als $\Delta S_k$.
        Dan is \[
            \sum_{k=1}^{n} f(\vec{p}_k)\Delta S_k
        \] een Riemann-som van $f$ bij partitie $P$.
\end{itemize}
\begin{opmerking}{}
    We kunnen de partitie zelf kiezen, er zijn oneindig veel Riemannsommen.
\end{opmerking}

We kunnen het volume beter afschatten met behulp van onder- en bovensommen:
\begin{align*}
    m_k = \inf\{f(\vec{x}) \mid \vec{x} \in D_k\} \implies O_D(f,P) = \sum_{k=1}^{n} m_k \Delta S_k \\
    M_k = \sup\{f(\vec{x}) \mid \vec{x} \in D_k\} \implies B_D(f,P) = \sum_{k=1}^{n} M_k \Delta S_k
\end{align*}
\[
    O_D(f,P) \leq \sum_{k=1}^{n} f(\vec{p}_k) \Delta S_k \leq B_D(f,P).
\]
Stel dat er precies één getal $V$ is zo dat voor elke partitie $P$ geldt dat
\[
    O_D(f,P) \leq V \leq B_D(f,P),
\] dan is $f(x,y)$ \textbf{Riemann-integreerbaar} over $D$.

Bekeken over alle partities:
\begin{itemize}
    \item $\{O_D(f,P) \mid P\, \text{partitie van}\, D\}$ is naar boven begrensd.
        Dan is het bezit van een supremum gelijk aan de \textbf{onderintegraal} $O_D(f)$.
    \item $\{B_D(f,P) \mid P\, \text{partitie van}\, D\}$ is naar onder begrensd.
        Dan is het bezit van een infimum gelijk aan de \textbf{bovenintegraal} $B_D(f)$.
\end{itemize}
Uit de definitie volgt algemeen
\[
    O_D(f) \leq B_D(f).
\]

\begin{definitie}{}{riemann-integreerbaar}
    $f(x,y)$ is \textbf{Riemann-integreerbaar} over $D$ indien
    \[
        O_D(f) = B_D(f).
    \]
    We noteren dit met $\displaystyle \iint_D f(x,y)\,dx\,dy$.
\end{definitie}

\begin{opmerking}{}
    In het boek gebruiken ze nog een andere formulering met $\epsilon$. Zie \Cref{def:double-integral}.
\end{opmerking}

\begin{definitie}{\cite[835]{cursus}}{double-integral}
    Een functie $f$ is \textbf{integreerbaar} over de rechthoek $D$ en heeft als \textbf{dubbele integraal}
    \[
        I = \iint_D f(x,y)\,dA
    \]
    als voor elk positief getal $\epsilon$ er een getal $\delta$ bestaat, zodat voor elke partitie $P$ van $D$ met $\lVert P\rVert < \delta$, en voor elke keuze van punten $(x_{ij}^*, y_{ij}^*)$ in de deelrechthoeken van $P$, geldt dat:
    \[
        \lvert R(f,P) - I\rvert < \epsilon,
    \]
    waarbij de Riemann-som gedefinieerd is als:
    \[
        R(f,P) = \sum_{i=1}^m \sum_{j=1}^n f(x_{ij}^*, y_{ij}^*) \Delta A_{ij}.
    \]
\end{definitie}

\printbibliography
\end{document}
